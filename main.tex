\documentclass{beamer}
\usepackage{tikz}
\usefonttheme{structuresmallcapsserif}
\usepackage{amsmath}
\usepackage[italian]{babel}
\usepackage{enumitem}
\usepackage{pifont}
\usepackage{xcolor}
\usetikzlibrary{calc,shapes}
\usepackage{hyperref}
\title{L'invasione della formica rossa di fuoco}
\author{Ilir Lika}
\institute{Istituto Maria Immacolata}
\date{\today}
\usetheme{CambridgeUS}
\logo{\includegraphics[height=1.4cm]{logo.jpg}}
\setbeamertemplate{caption}{\insertcaption}
\begin{document}
\frame{\titlepage}
	\section{indice}
\begin{frame}{indice}
	\begin{itemize}[label=$\bullet$]
		\item introduzione
		\item il nostro protagonista
   \begin{itemize}[label=\ding{212}]
	\item di chi stiamo parlando?
   \item un ritratto della formica
   \begin{itemize}[label=$\blacksquare$]
       \item la necrophoresis
   \end{itemize}
   \end{itemize}
		\item da dov'è arrivata?
        \item com'è arrivata in Italia?
        \item le previsioni
        \item cosa possiamo fare?
    \end{itemize}
	\end{frame}
\begin{section}{introduzione}
\begin{frame}{introduzione}
    \begin{columns}
    \column{0.4\textwidth}
        \centering
        oggi in Italia si può assistere ad \alert{una vera e propria invasione} di una specie invasiva(non autoctona) che già soltanto negli USA fa spendere
        \begin{alertblock}{ogni anno}
        più di 5 miliardi di dollari 
        \end{alertblock}
    \column{0.6\textwidth}
        \centering
        \includegraphics[scale=0.4]{invasione.jpg}
    \end{columns}
    
\end{frame}
\end{section}
\begin{section}{il nostro protagonista}
\begin{frame}{di chi stiamo parlando?}
\begin{columns}
\column{0.5\textwidth}
    \centering
    Stiamo parlando della formica rossa di fuoco(\alert{Solenopsis invicta})
    \includegraphics[width=0.9\textwidth]{solenopsis.jpg}
\column{0.5\textwidth}
    \centering 
    da non confondere con la più comune,presente anche qui in Piemonte,formica rossa(\alert{Formica rufa}) \\
    \includegraphics[width=0.6\textwidth]{rufa.jpg}
\end{columns}
\end{frame}
\begin{frame}{un ritratto della formica}
La RIFA(red imported fire ant) oltre ad essere un pericolo per l'economia e la salute pubblica è anche 
\begin{block}{una formica}
incredibilmente particolare per molti aspetti
\end{block}
\begin{itemize}[label=\ding{212}]
\item è sia poliginica che monoginica
\item è polimorfa
\item è velenosa
\item è molto eusociale e presenta comportamenti molto particolari
\end{itemize}
\end{frame}
\begin{frame}{poliginia e monoginia}
    è sia poliginica che monoginica,ovvero le società possono essere fatte in alcuni casi(monoginia) da \alert{una sola regina} ed in altri(poliginia) da \alert{più regine},questa caratteristica viene determinata da un singolo allele,questo passaggio da un allele all'altro è spiegato molto bene \color{blue}{\href{https://cordis.europa.eu/project/id/623713/results/it}{in degli articoli fruibili nel CORDIS}} \\
    \includegraphics[width=0.5\textwidth]{regina.jpg}

\end{frame}
\begin{frame}{polimorfismo}
 è polimorfa, la società o,ancora meglio,il super organismo,dove le sue cellule sono le formiche stesse, è divisa in caste e queste sono determinate geneticamente creando formiche che devono compiere ruoli diversi in modi diversi(le operaie sono piccole ed economiche da produrre,i soldati sono grandi e forti etc…) 

\includegraphics[width=0.6\textwidth]{polimorfismo.png}
\end{frame}
\begin{frame}{è velenosa}
    è molto velenosa,questo l'ha resa,in funzione di preservare la salute pubblica: \\ \textit{\alert{la specie di formica finora studiata in modo più approfondito}, si serve di circa \alert{20 segnali} }(per comunicare)\textit{: la cifra esatta dipende da quali funzioni – tra quelle molto simili – si vogliono considerare distinte. \alert{Solo due segnali sono tattili, tutti gli altri sono chimici}}[Il superorganismo]
    \includegraphics[scale=0.2]{sting.png}
\end{frame}
\subsection{eusocialità}
\begin{frame}{è molto eusociale}
    siccome questo è l'aspetto che mi interessa maggiormente ho scelto dei particolari comportamenti che spero troverete anche voi affascinanti,quelli che ho scelto sono :
    \begin{itemize}[label=\ding{212}]
        \item rilascio di una goccia di veleno sul pungiglione proseguito dal suo agitamento(flagging) per spaventare i nemici
        \item costruzione di zattere quando si alza il livello dell'acqua
        \item presenza della \textit{necrophoresis}
    \end{itemize}
     \begin{figure}
      \begin{columns}
        \column{.4\linewidth}
        \includegraphics[]{raft.jpg}
        \column{.1\linewidth}
        \rotatebox[origin=c]{180}{\ding{212}}
        \column{.5\linewidth}
        \caption{ zattera tipica formata da S. invicta }
      \end{columns}
    \end{figure}
      %  \includegraphics[scale=1.0]{raft.jpg}
\end{frame}
\begin{frame}{necrophoresis}
\begin{columns}
    \column{0.6\textwidth}
    \centering
     questo comportamento,\alert{presente esclusivamente negli insetti sociali}(api,formiche,etc...),consiste nella sanificazione del nido attraverso la \alert{rimozione dei cadaveri} per evitare la diffusione di potenziali virus,questo comportamento scoperto dal biologo E.O. Wilson risulta ereditario e viene \alert{fatto spesso da solo delle classi specializzate}.É stato molto importante perchè ha permesso al biologo di spiegare come \alert{comportamenti e tratti della personalità possano essere determinati anche dai geni},permettendo l'invenzione della sociobiologia
    \column{0.4\textwidth}
    \includegraphics[scale=0.4]{Necrophoresis.jpg}
\end{columns}

\end{frame}
\end{section}
\begin{frame}{da dov'è arrivata?}
    la RIFA è nativa del Sud America,però,proprio perchè molto resiste,quasi invincibile(da qui "invicta"),è riuscita ad espandersi in \alert{gran parte del mondo},ecco una mappa interattiva \\
    \href{https://antmaps.org/?mode=species&species=Solenopsis.invicta}{\includegraphics[scale=0.12]{map.png}} 
\end{frame}
\begin{frame}{com'è arrivata in Italia?}
\begin{columns}
    \column{0.4\textwidth}
      \begin{block}{domanda}
            com'è possibile che siano stati trovati ben 88 nidi di S. invicta a Siracusa?
    \end{block}
      \begin{alertblock}{risposta}
        è indubbiamente collegato al fatto che la città sia in prossimità di un porto dentro il quale saranno arrivate delle navi contenenti ancora qualche formica viva
    \end{alertblock}
    \column{0.6\textwidth}
    \includegraphics[scale=0.15]{siracusa.jpg}
\end{columns}
\end{frame}
  \begin{frame}{le previsioni}
      \begin{columns}
            \column{0.4\textwidth}    
                \includegraphics[scale=0.1]{previsioni.png}
            \column{0.6\textwidth}
            {\tiny
            (A) Map showing occurrence data (red dots) and background (blue areas) used
for species distribution models of S. invicta. (B) Ensemble model predictions under current environmental conditions at a
global scale. (C) Ensemble model predictions for future environmental scenarios at a global scale for the year 2050. (D)
Ensemble model predictions for future environmental scenarios at a global scale for the year 2070. (E) Ensemble model
predictions for future environmental scenarios at a global scale for the year 2090. (F) Ensemble model predictions for future
environmental scenarios at European and Mediterranean scale for the year 2070 and (G) for the year 2090. (H) Binary
ensemble model predictions under current environmental conditions and (I, J, K) for future environmental scenarios at
European and Mediterranean scale.
}
      \end{columns}
  \end{frame}

\begin{frame}{cosa possiamo fare?}
    Siccome in tutta la storia di spese e tentativi di eradicazione della specie solo in un caso si ha avuto successo(in Nuova Zelanda) penso sia meglio accettare l'inevitabile,ma comunque preoccuparsi della \alert{salvaguardia della biodiversità} che viene spesso messa a rischio da specie invasive,come anche il granchio blu,per il semplice fatto che nel nuovo habitat non esistono ancora parassiti specifici per loro e tendenzialmente gli esemplari che riescono a sopravvivere al viaggio così tanto lungo sono molto resistenti.Inoltre,come si evince dai grafici,\alert{prevenire il più possibile il cambiamento climatico} avrebbe un'ottimo impatto 
\end{frame}

\end{document}
