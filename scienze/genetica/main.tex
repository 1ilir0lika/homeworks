\documentclass[a4paper]{article}
\usepackage[margin=2cm]{geometry}   

\usepackage{paracol} % added <<<<<<<<<<<<<<<<<<<<<
\usepackage{tcolorbox}

\usepackage{lipsum}% generate filler text 

\newcommand{\question}[1]{% Ask the question
    \begin{tcolorbox}[colback=red!05,colframe=red!25,sidebyside align=top,width=\linewidth,before skip=1ex]
        #1
        
    \end{tcolorbox}%    
    \switchcolumn % now write in the right column   
}

\newcommand{\note}[1]{% Add as many notes as you like
    \begin{tcolorbox}[colback=blue!05,colframe=blue!10,width=\linewidth,before skip=1ex]
        #1
    \end{tcolorbox}         
}   

\newcommand{\summary}[2][]{%
\begin{minipage}[b]{\textwidth}
    \vspace*{\baselineskip}
    \begin{tcolorbox}[colframe=green!50!blue!80,fonttitle=\large\bfseries\sffamily,
        after skip = \baselineskip,
        title=Riassunto]
        #2
    \end{tcolorbox}
\end{minipage}
#1}


\setcolumnwidth{0.40\textwidth/20pt,0.60\textwidth}% column separation =20pt
\setlength{\columnseprule}{3pt} % column width
\colseprulecolor{red}

\title{%
        \begin{tcolorbox}[before skip = -\baselineskip, after skip =-\baselineskip]
            \centering\Huge\sffamily Il DNA   
        \end{tcolorbox}
}

\date{}
\parindent=0pt

\begin{document}
    
\maketitle  

\section{Introduzione}

\begin{paracol}{2}
\question{Che cos'é Il DNA?}
\note{Il DNA contiene tutte le istruzioni necessarie per sintetizzare le cellule di un dato organismo}
\note{é come una lingua universale fatta solo dalle lettere A,T,G,C}
\note{é a forma di scala a chiocciola}
\note{per comprendere questa lingua si sfrutta il \textbf{codice genetico} che associa ad ogni sequenza di 3 lettere(un codone) un amminoacido $\rightarrow$ unitá delle proteine}
\end{paracol}

\begin{paracol}{2}  
    \question{Dov'é contenuto il DNA?}
    \note{\begin{itemize}
    \item procarioti $\rightarrow$ filamenti circolari di DNA
    \item eucarioti $\rightarrow$ cromosomi
\end{itemize}
\includegraphics[width=0.7\textwidth]{Karyotype.jpg}\\
foto del cariotipo umano dove si possono vedere i suoi cromosomi
}
\end{paracol}
\begin{paracol}{2}
\question{Che cos'é il genoma?}
\note{\textit{L'intera informazione genetica che caratterizza ogni organismo vivente, codificata dal DNA si chiama \textbf{genoma}}}
\end{paracol}
\begin{paracol}{2}
    \question{Complessitá di un organismo}   
    \note{il numero di cromosomi non é associato ad una maggiore complessitá di un organismo,una cipolla ha piú di cinque volte il nostro DNA nelle proprie cellule}
\end{paracol}
\begin{paracol}{2}
    \question{Che cos'é un gene?}
    \note{Un gene é una sezione di DNA che codifica per un prodotto funzionale(generalmente una \textbf{proteina})}
    \note{le diverse possibili "versioni" dello stesso gene si chiamano \textbf{allele}}
    \note{ogni caratteristica di un organismo si chiama \textbf{tratto}}
\end{paracol}
\summary{Il DNA contiene tutte le informazioni di un organismo vivente.Viene organizzato in cromosomi}
\section{Non tutto il DNA contiene istruzioni per sintetizzare proteine}
\begin{paracol}{2}
\question{Junk DNA}
\note{nell'uomo solo il 2\% del DNA é composto da geni,il resto non ha alcuna funzione $\rightarrow$ \textbf{junk DNA}}
\note{\begin{itemize}
    \item procarioti $\rightarrow$ quasi assente
    \item eucarioti $\rightarrow$ piú del 90\% del DNA
\end{itemize}
\includegraphics[scale=1.4]{The-percentage-of-protein-coding-genes-sequences-in-several-eukaryotic-and-bacterial.png}\\
un diagramma con le percentuali del DNA codificante in diversi organismi
}
\note{il 25\% dei geni non codificanti si trovano dentro altri geni $\rightarrow$ \textbf{introni}}

\end{paracol}
\begin{paracol}{2}
\question{perché esiste?}
\note{puó servire a permettere a delle sequenze del DNA di fare copie di se stesse e di muoversi nel genoma}
\note{sono frammenti di geni o pseudogeni che si sono formati da dei geni veri,ma accumulando troppe mutazioni non sono piú in grado di codificare proteine}
\note{non si é certi delle loro funzioni}
\note{recentemente si ha riscontrato che sintetizzano porzioni piccole di RNA che fungono da "switch" per regolare l'espressione genica}
\note{potrebbe anche essere una riserva per delle sequenze potenzialmente utili}
\end{paracol}
\summary[\clearpage]{la maggior parte del nostro genoma é fatto da junk DNA,questo non sintetizza alcuna proteina}
\section{Come funzionano i geni?}
\begin{paracol}{2}
        \question{genotipo}
        \note{i geni che codificano per un particolare tratto}
\end{paracol}
\begin{paracol}{2}
        \question{fenotipo}
        \note{la manifestazione del genotipo}
\end{paracol}
\begin{paracol}{2}
        \question{dal genotipo al fenotipo}
        \note{richiede 
        \begin{itemize}
            \item \textbf{trascrizione} $\rightarrow$ mRNA
            \item \textbf{traduzione} $\rightarrow$ polipeptide
        \end{itemize}
        \includegraphics[width=\textwidth]{trascrizione traduzione.jpg} \\
        ecco un'illustrazione che riassume i 2 processi
        }
\end{paracol}
\summary{affinché si possa passare dal genotipo al fenotipo é necessaria la trascrizione e la traduzione}
\end{document}
