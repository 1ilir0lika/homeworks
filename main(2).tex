%%%%%%%%%%%%%%%%%
% This is a two-column CV template based on moderncv.cls
% (v1.0, May 16 2024) written by Genki Ogaki (rockstarogk@gmail.com). Compiles with XeLaTeX.
%
%% It may be distributed and/or modified under the
%% conditions of the LaTeX Project Public License, either version 1.3
%% of this license or (at your option) any later version.
%% The latest version of this license is in
%%    http://www.latex-project.org/lppl.txt
%% and version 1.3 or later is part of all distributions of LaTeX
%% version 2003/12/01 or later.
%%%%%%%%%%%%%%%%

% environment
\documentclass[11pt,a4paper]{moderncv}
\moderncvtheme[blue]{banking}
\nopagenumbers{}
\usepackage{hyperref}
\hypersetup{
    colorlinks=true,
    linkcolor=blue,
    filecolor=magenta,      
    urlcolor=blue,
    pdftitle={Overleaf Example},
    pdfpagemode=FullScreen,
    }
\usepackage[T1]{fontenc}
\usepackage{inputenc}
\usepackage[scale=0.9]{geometry}
\usepackage{tabularx}
\usepackage{mathpazo}

% macro
\renewcommand*{\labelitemi}{-}

\newcolumntype{L}{>{\raggedright\arraybackslash}X}
\newcolumntype{C}{>{\centering\arraybackslash}X}
\newcolumntype{R}{>{\raggedleft\arraybackslash}X}

\newcommand*{\experienceentry}[3][1.5mm]{
    \subsection{#2} \vspace{-1.5mm}
    \begin{tabularx}{\textwidth}{LR}
        {\itshape #3}
    \end{tabularx}
    \par\addvspace{#1}
}

\newcommand*{\educationentry}[4][0.5mm]{
    \begin{tabularx}{\textwidth}{LR}
        {\bfseries #3} & {\bfseries #4} \\
    \end{tabularx}
    {\itshape #2}
    \par\addvspace{#1}
}

\newcommand*{\scoreentry}[3][2.5mm]{
    {\bfseries #2} \\
    {\itshape #3}
    \par\addvspace{#1}
}

% preamble
\firstname{Ilir}
\familyname{Lika}
\address{Istituto Maria immacolata,Pinerolo}{}

% document
\begin{document}
\maketitle
\vspace{-9.0mm}
\begin{tabularx}{\textwidth}{C}
    \emailsymbol\enspace \emaillink{ilir.lika@istitutomariaimmacolata.it} 
\end{tabularx}
\vspace{-2.0mm}

% left column
\begin{minipage}[t]{0.62\textwidth}
\section{PROGETTI PERSONALI}
\experienceentry{Animazioni con Manim}{dal 2022}

\textbf{Animazioni ed analisi dati mediante linguaggio python}
\begin{itemize}
    \item \href{https://youtu.be/O17m9bKEgNs?feature=shared}{Teorema di Bayes}
    \item \href{https://youtu.be/3SJeTFr3EhY?feature=shared}{Funzioni}
    \item \href{https://github.com/1ilir0lika/video-con-manim}{Molti altri}
\end{itemize}
\vspace{1.0mm}

\experienceentry{Stazione metereologica}{luglio 2022}

\textbf{Elettronica e analisi dati}
\begin{itemize}
    \item Progettato e realizzato circuito Arduino per la misurazione della temperatura,pressione ed umiditá atmosferica
    \item Sviluppato bot telegram per notificarmi delle temperature esterne
    \item Analizzato questi dati e rappresentati mediante diagrammi climatici utilizzando ggplot2 \\
    \includegraphics[width=0.3\textwidth]{climatico.pdf}
\end{itemize}
\vspace{2.0mm}

\experienceentry{Scraper con selenium}{2023}

\textbf{Web scraping,analisi dati e data visualization}
\begin{itemize}
    \item Sviluppato web scraper in grado di accedere ai prezzi di un sito web e salvarli in un file con le rispettive date
    \item Sviluppato script in python in grado di analizzare l'andamento dei prezzi per consigliare cosa comprare
\end{itemize}
\vspace{1.0mm}

\experienceentry{Sondaggi di gruppo su calendario}{Giugno 2024}

\textbf{database SQL}
\begin{itemize}
    \item Sviluppato bot telegram in grado di far inserire i giorni nei quali non si è disponibili e salvarli in un database 
    \item analisi del database per capire i giorni dove più persone sono disponibili per il dato evento  \\
    \includegraphics[width=0.5\textwidth]{bot.png}
\end{itemize}
\end{minipage}
\hfill
% right column
\begin{minipage}[t]{0.35\textwidth}
\section{RIASSUNTO}
Studente di terza superiore affascinato dalle materie scientifiche che costantemente cerca di apprendere nuove conoscenze.

\section{DATI ANAGRAFICI}
\begin{tabularx}{\textwidth}{>{\bfseries}l@{\hskip 3.5mm}L}
Nome & Ilir \\
Cognome & Lika \\
Data nascita & 21/02/2007 \\
\end{tabularx}
\section{SKILLS}
\begin{tabularx}{\textwidth}{>{\bfseries}l@{\hskip 3.5mm}L}
Back End & Rust,Python,Perl \\
Data visualization & Panda,\href{https://www.manim.community/}{Manim},Ggplot2 \\
OS & Linux, Bsd\\
DevOps & Docker\\
Misc. & \LaTeX,Vim \\
Soft Skills & Problem solving
\end{tabularx}

\section{ISTRUZIONE}
\educationentry{elementari e medie}{Istituto comprensivo Franco Marro}{Villar Perosa, Torino}
\par
\vspace{3.0mm}
\educationentry{superiori}{Istituto Maria Immacolata}{Pinerolo, Torino}

\section{PUNTEGGI}
\scoreentry{Olimpiadi chimica triennio (2023/24)}{57/114 classe B}
\scoreentry{Olimpiadi scienze biennio}{30/81 classe B}
\scoreentry{First}{Overall: 175}
\section{LINGUE}
\begin{itemize}
    \item Italiano (nativo)
    \item Inglese (B2)
\end{itemize}

\section{HOBBY}
\begin{itemize}
    \item Scacchi
    \item Leggere libri non di narrativa
\end{itemize}
\end{minipage}

\end{document}
